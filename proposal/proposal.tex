\documentclass[11pt]{article}
\bibliographystyle{siam}

\title{The generality of self-control}
\author{
  Udagawa, Hiroto\\
  \texttt{hiroto-udagawa}
  \and
  LeRoy, Benjamin\\
  \texttt{benjaminleroy}
  \and
  Lee, Rachel\\
  \texttt{reychil}
  \and
  Liang, Jane\\
  \texttt{janewliang}
  \and
  Chen, Kent\\
  \texttt{kentschen}
}

\begin{document}
\maketitle

\section{Summary:}

We plan to study the curated OpenfMRI data set and paper entitled: \textit{The generality of self-control} \cite{lindquist2008statistical}. After downloading the data, we were able to load the data set and confirm that it has the correct number of subjects. This paper and its associated fMRI studies are concerned with the relationship between impaired and normal self control as well as the similarities and differences across the brain relating to self-control. The entire paper explores multiple studies (of multiple study types), but we will just focus on the third and final study, which compares four different types of self control among healthy adults to see if they are related to each other. These four forms of self-control all require inhibition of a prepotent response in order to either not respond or to make the correct response: (1) motor control (stop-signal tasks); (2) control over risk-taking behavior (balloon analog risk tasks); (3) control to be able to delay gratification (temporal discounting); and (4) emotional control (emotional regulation. Participants were paid to encourage cooperation with the studies. A great deal of black box preprocessing was done on the collected fMRI data to eliminate faulty trials and noise. Very little relationship was found between these behavioral tasks, in contrast to the vast majority of existing literature, which argues for a unified notion of self-control.

\section{Approach:}
Our approach for exploring the data is primarily reproduction of the paper’s analytical approaches, with some further statistical analyses, hopefully with additional results and findings. We found that the many of the paper’s analytical approaches were not especially sophisticated, and involved simple procedures like ANOVA contrasts. While most of the data cleaning was done using packaged software that we may not be able to reproduce easily, we thought it would be possible to reproduce most of the analytical processes and also to augment the analyses with our own ideas, including concepts from machine learning. For example, we could easily consider using other correlation methods besides the Pearson correlation, especially since Pearson correlation is not robust for non-linear relationships with non-normally distributed data. Likewise, the paper uses Gaussian kernel smoothers, and we could consider alternative kernels or methods for spatial smoothing. We could also use different corrections for handling multiple testing to see if we can draw different conclusions from the data. The paper largely considers only one “snapshot” in time per participant per study, which is interesting, but we could explore using more than one “snapshot”, thereby increasing our feature space. Some of our more complex ideas may be too computationally intensive for data of this size, but parallelization and possibly using subsets of the data may go a long way for making such procedures feasible. 



We plan to study the curated OpenfMRI data set and paper entitled: "The generality of self-control". After downloading the data, we were able to load the data set and confirm that it has the correct number of subjects. This paper and its associated fMRI studies are concerned with the relationship between impaired and normal self control as well as the similarities and differences across the brain relating to self-control. The entire paper explores multiple studies (of multiple study types), but we will just focus on the third and final study, which compares four different types of self control among healthy adults to see if they are related to each other. These four forms of self-control all require inhibition of a prepotent response in order to either not respond or to make the correct response: (1) motor control (stop-signal tasks); (2) control over risk-taking behavior (balloon analog risk tasks); (3) control to be able to delay gratification (temporal discounting); and (4) emotional control (emotional regulation. Participants were paid to encourage cooperation with the studies. A great deal of black box preprocessing was done on the collected fMRI data to eliminate faulty trials and noise. Very little relationship was found between these behavioral tasks, in contrast to the vast majority of existing literature, which argues for a unified notion of self-control.

\section{Approach:}
Our approach for exploring the data is primarily reproduction of the paper’s analytical approaches, with some further statistical analyses, hopefully with additional results and findings. We found that the many of the paper’s analytical approaches were not especially sophisticated, and involved simple procedures like ANOVA contrasts. While most of the data cleaning was done using packaged software that we may not be able to reproduce easily, we thought it would be possible to reproduce most of the analytical processes and also to augment the analyses with our own ideas, including concepts from machine learning. For example, we could easily consider using other correlation methods besides the Pearson correlation, especially since Pearson correlation is not robust for non-linear relationships with non-normally distributed data. Likewise, the paper uses Gaussian kernel smoothers, and we could consider alternative kernels or methods for spatial smoothing. We could also use different corrections for handling multiple testing to see if we can draw different conclusions from the data. The paper largely considers only one “snapshot” in time per participant per study, which is interesting, but we could explore using more than one “snapshot”, thereby increasing our feature space. Some of our more complex ideas may be too computationally intensive for data of this size, but parallelization and possibly using subsets of the data may go a long way for making such procedures feasible. 

\bibliography{proposal}



\end{document}
