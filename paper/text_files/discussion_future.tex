% tex file for future discussion

\par \indent To extend our work on multiple linear regression, we will 
explore methods to account for more of the noise in our data, such as linear 
drift and discrete cosine transforms for general trends (similar to Fourier 
series as extra features). We will also be looking into (as long as this is 
not corrected with pre-processing from out mentors) realignment of scans to 
correct for the time it takes to scan each voxel compared to the start of the 
scan. One approach to correct the time shift of when the voxels were actually 
scanned is to perform resampling techniques.

\par Almost all of our currently implemented methods are designed to handle 
each subject's data individually. However, we would like to take advantage of 
all 24 subjects, and would like to explore procedures for comparing results 
across subjects or aggregating subject data together. 

\par One major concern for hypothesis testing and working with the estimated 
coefficients from linear modeling is the issue of multiple comparisons. This 
may be as simple as utilizing a Bonferroni correction, but we can also 
consider permutation tests or more sophisticated approaches (random fields, 
Benjamini-Hoffberg, etc). 

\par More quantitative and robust indicators for validating time series 
models should be implemented. One possibility is to simulate a null process 
by permuting the phases of the voxel’s time course after performing a Fourier 
transform, and then transforming the permuted data back to the original 
space. This way, the data is permuted but maintains the original correlated 
structure. We can then fit the same ARIMA model to the permuted process and 
examine how much, if at all, the ARIMA process fitted to the observed data 
makes improvements over the null case. Generating confidence intervals for 
the parameter estimates and forecasting future observations may also be of 
interest. 
