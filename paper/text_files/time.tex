% tex file for time series
\par \indent Cohen's paper \cite{CohenSelfControl} discusses analyzing the data with time series using FILM (FMRIBs Improved Linear Model). While we are not familiar with the FILM method, we did try modeling individual voxels in the framework of an autoregressive integrated moving average (ARIMA) process. We focused only on a single voxel from the first subject, but the method could easily be extended to additional or aggregate voxels. Let $\{Y_t\}$ be a single volume's value at time $t$ and assume that the $d$th difference $W_t = \nabla^d Y_t$ is weakly stationary, defined to be when $W_t$ has a constant mean function and autocovariance dependent only on lag $k$ and not time $t$. Then we can try to model $W_t$ as a linear combination of $p$ autoregressive terms (or the number of most recent values to include) and $q$ moving average terms (the number of lags to include for the white noise error terms): 
$$W_t = \phi_1 W_{t-1} + \phi_2 W_{t-2} + ... + \phi_p W_{t-p} + e_t - \theta_1 e_{t-1} - \theta_2 e_{t-2} - ... - \theta_q e_{t-q}.$$

\par White noise is defined as a sequence of independent, identically distributed random variables. In order to fit an ARIMA process, the three orders $p$, $d$, and $q$ must be first be specified, and the the associated coefficients estimated. We used a combination of visual inspection and quantitative methods to specify the ARIMA orders, and then used the maximum likelihood method to estimate parameters. 
