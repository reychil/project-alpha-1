% tex file for introduction
\par \textit{The Development and Generality of Self-Control} \cite
{CohenSelfControl} and its associated fMRI studies are concerned with the 
relationship between impaired and normal self control, as well as 
similarities and differences across the brain relating to self-control. The 
paper in its entirety explores multiple studies (of multiple study types), 
but we will just focus on the third and final study, which compares four 
different types of self control among healthy adults to see if they are 
related to each other. Very little relationship was found between these 
behavioral tasks, in contrast to the vast majority of existing literature, 
which argues for a unified notion of self-control. So, we have decided to 
narrow our focus and data analysis approaches to just the Balloon Analogue 
Risk Task (BART) study, which purportedly measures control over risky 
behavior. fMRI scans from the study show blood flow to the brain, which may 
be relatable to control over risk-taking behavior during participation.

\par The rest of this report will detail the procedures from the original 
analysis of the data that we have tried to mimic. We looked into spatial 
smoothing of the data on the voxels of the brain. After obtaining convolved 
time courses for each subject, we turned to fitting simple and multiple 
linear regression models to each subject, with the option of including study 
conditions. Since examining the behavior of blood flow in the voxels over 
time was of such great interest, we also considered modeling the behavior as 
a time series using an autoregressive integrated moving average (ARIMA) 
process. 
